\pagenumbering{arabic}
\setcounter{page}{1}
\titleformat{\chapter}[display]
    {\normalfont\Large\bfseries}{\filleft\chaptertitlename\ \thechapter}{20pt}{\Huge}    
\chapter{Introducción}
    El problema del vendedor viajero, también conocido como \textbf{TSP} (\textit{travelling salesman problem}), ha sido objeto de estudio durante décadas. Sus orígenes se remontan al siglo \textbf{XX}, y gracias a destacados personajes históricos y avances tecnológicos en el campo de la computación, se han logrado importantes avances en la búsqueda de soluciones para los problemas que aún persisten en la actualidad. Casos como el \textit{problema de la última milla} y el \textit{problema de ruteo de vehículos} han encontrado respuestas parciales a través de este extenso camino de investigación.
    \newline
    \newline
    El \textbf{TSP} se refiere a la optimización de una ruta que debe seguir un vendedor para visitar un conjunto de destinos, buscando la ruta más corta y eficiente posible. Este problema se ha abordado desde diferentes perspectivas, tanto en el ámbito de la programación como en el campo matemático, con el objetivo de encontrar soluciones que minimicen los costos y maximicen la eficiencia en la entrega o visita de los destinos.
    \newline
    \newline
    A lo largo de este trabajo, se explorarán diferentes enfoques y metodologías para abordar el \textbf{TSP}, incluyendo técnicas heurísticas, algoritmos de optimización y enfoques basados en metaheurísticas. Además, se analizarán casos reales en los que el TSP ha encontrado aplicaciones prácticas, como en el \textit{problema de la última milla} en el comercio electrónico, el \textit{ruteo de vehículos} en la logística y el transporte, y la \textit{programación de robots} para tareas específicas.
    \newline
    \newline
    El objetivo principal es comprender la importancia y las implicaciones del \textbf{TSP} en la resolución de problemas complejos en diversos campos, así como examinar las soluciones y avances alcanzados hasta la fecha. A través de este estudio, se espera contribuir al conocimiento y la comprensión de esta problemática, y su potencial para generar mejoras significativas en la eficiencia y rentabilidad de distintas industrias.