\chapter{Conclusiones}
\justify
    Dentro de la investigación realizada se han propuesto distintos enfoques para trabajar \textbf{TSP,} entre ellos destaca principalmente el uso de metaheurísticas, de esta manera problemas enormes como el \textit{traveler salesman problem} se reducen a un conjunto mas acotado de posibles soluciones. De este modo, como se pudo comprobar dentro del capítulo de aplicación de metaheurísticas, utilizar estas herramientas permiten transformar el problema original en un derivado un poco más maleable, es importante recordar que el campo de aplicación de estas depende casi en su totalidad del problema con el que se esta trabajando. Para el caso que se presenta en el documento, depende en gran parte que utilidad se quiere dar al \textbf{TSP}, por ejemplo, para \textbf{ACO} trabajar con pocas “ciudades” es contraproducente puesto que \textbf{ACO} consume una gran cantidad de recursos para recordar las rutas anteriores. Por su parte \textbf{GRASP}, si bien es mas eficiente que \textbf{ACO} su principal implementación esta dentro de los cortes de materiales dentro de la industria. En general, comparar estas metaheurísticas mediante pruebas y estadísticas permitió ver sus debilidades y fortalezas.
    \newline
    \newline
    Por su parte, trabajar este problema como un problema de optimización clásico en \textbf{Gurobi} permitió llevar el problema a un caso de análisis “real” en el que se comienzan definiendo variables y restricciones para luego proponer heurísticas o metaheurísticas para una posible solución. En definitiva, el trabajo realizado con \textbf{Gurobi} permite visualizar el problema como caso de industria “real”, permitiendo plantear correctamente el problema, previo a las soluciones posibles.
    \newline
    \newline
    Por último, el análisis de los casos de implementación reales de \textbf{TSP} permitieron hacer un acercamiento al estado del arte del problema en cuestión. Es más, gracias a este capitulo se puede apreciar lo importante que ha sido el problema del vendedor viajero a lo largo de la historia, permitiendo analizar problemas que, sin él, su entendimiento seria aun mas complejo. De esta manera, a continuación, se puede reducir aún más la problemática mediante heurísticas y metaheurísticas, que, si bien no reducen la complejidad del problema, es decir, sigue siendo \textbf{NP completo}, reduce su eficiencia y eficacia de todos modos.
    